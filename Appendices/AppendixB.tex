% !TEX root = ../main.tex
% Appendix A

\chapter{Language feature categorization} % Main appendix title

\label{AppendixB} % For referencing this appendix elsewhere, use \ref{AppendixA}

\lhead{Appendix B. \emph{Language feature categorization}} % This is for the header on each page - perhaps a shortened title

The following appendix contains a categorization of ECMAScript 6 language features according to the taxonomy established in chapter \ref{taxonomy}

\section{Arrow Functions} \label{arrow}
Arrow functions\cite[14.2]{SpecJS} are the new lambda-like function definitions inspired by the Coffeescript\footnote{\url{http://coffeescript.org/}} and C\# notation. The functions body knows no lexical this binding (see section \ref{javascript-scoping}) but instead uses the binding of its parent lexical scope.

\paragraph{Category}
The arrow function is eliminated through transformation to a normal JavaScript function expression wrapped in a IIFE to bind \textit{this} and \textit{arguments} from the enclosing lexical scope. Arrow functions thus categorize as syntactic sugar.

\paragraph{Scope}
To avoid a \textit{ReferenceError} when an arrow function is used outside of the lexical scope of a function, the \textit{\_arguments} identifier should reference \textit{undefined}. This is not the case when an arrow function is used inside the lexical scope of a function, then \textit{\_arguments} references the value of \textit{arguments} in its containing lexical function. This implies that the transformation of an arrow function requires context information of the placement of this arrow function, thus the transformations scope is global-to-local.

\paragraph{Analysis of sub-terms}
References to the \textit{arguments} and \textit{this} keyword have to be identified in the body of an arrow function and renamed.
\\
\\
\blockquote[{\cite[14.2.16]{SpecJS}}]{An ArrowFunction does not define local bindings for \textbf{arguments}, \textbf{super}, \textbf{this}, or new.target. Any reference to arguments, super, or this within an ArrowFunction must resolve to a binding in a lexically enclosing environment.}
\\\\
An arrow function is transformed to an ES5 function. This function is wrapped in a immediately invoking function expression, this IIFE is used to create a new closure with the \textbf{this}, and \textbf{arguments} bindings of the lexical enclosing scope bound to \textbf{\_this} and \textbf{\_arguments} respectively. The enclosing lexical scope is not polluted with new variable declarations. All references to \textit{this} and \textit{arguments} inside the body of arrow function are renamed to the underscored names (only references in the current scope (i.e. no deeper scope) are renamed).

\begin{lstlisting}
var f = <Arrow Function>;
\end{lstlisting}

\begin{lstlisting}
var f = (function(_arguments,_this) {
	return <Desugared Arrow Function>
})(arguments,this);
\end{lstlisting}

\paragraph{Constraints on sub-terms}
Use of the \textbf{yield} keyword is not allowed in an arrow functions body. Arrow functions can not be generators (see appendix \ref{generators}) and deep continuations are avoided (i.e. yielding from inside an arrow function to the lexical enclosing generator).

\paragraph{Preconditions}
Before an arrow function can be transformed its body should have been transformed. This to prevent the incorrect renaming of sub-terms (\textbf{super}, \textbf{arguments}, \textbf{this}), because they still fall in the same scope depth. When visiting the body of an arrow function to rename sub-terms the visitor breaks a path once an ES5 function is found, it does not break a path once it finds an arrow function. The following example illustrates how a desugaring would fail to correctly rename sub-terms if the body is not already desugared.

\begin{lstlisting}
var f = () => {
	() => this;
};
\end{lstlisting}

\begin{lstlisting}[caption={Incorrect renaming of this in nested arrow function}]
var f = (function(_this,_arguments) {
	return function() {
		() => _this;
	};
})(this,arguments);
\end{lstlisting}

\section{Classes}
Class definitions~\cite[14.5]{SpecJS} are introduced in ECMAScript 6 as a new feature to standardize the inheritance model.
This new style of inheritance is still based on the prototypal inheritance (see section \ref{javascript-inheritance}).

\paragraph{Category}
The class construct is syntactic sugar for simulation of class based inheritance through the core prototypal inheritance system of the ECMAScript language.

\paragraph{Syntactically type preserving}
There are two types of class declarations, a direct declaration as a Statement, or an Expression declaration. They are both transformed to a construct of the same syntactical type.

\paragraph{Depending on bindings}
Several features of the class declaration as described in the specification use some run-time functions. This to avoid unnecessary "noise" in the generated program. The following features use a run-time function

\begin{itemize}
	\item \textit{(\_classCallCheck)}: Check if constructor was called with \textit{new}
	\item \textit{(\_createClass)}: Add method as non-enumerable property to prototype of class.
	\item \textit{(\_inherits)}: Inherit from the prototype of the parent class.
	\item \textit{(\_get)}: Retrieve properties or constructor of the parent class.
\end{itemize}

\paragraph{Analysis of sub-terms}
References of \textbf{super} and a \textbf{super} call have to be identified inside the constructor or class methods. These are transformed to preserve the correct binding of \textbf{this}.

\paragraph{Constraints on sub-terms}
No.

\paragraph{Preconditions}
Bodies of constructor and class methods should have been transformed before the class declaration itself is transformed. This to prevent incorrect transformation of the sub-term \textbf{super}. (See preconditions of arrow functions, this is the same problem)

\section{Destructuring} \label{destructuring}
Destructuring~\cite[12.14.5]{SpecJS} is a new language construct to extract values from object or arrays with a single assignment. It can be used in multiple places among which parameters, variable declaration, and expression assignment.

\paragraph{Category}
The destructuring language feature is eliminated by translating it into core language concepts such as member access on objects and array member selection.

\paragraph{Syntactically type preserving}
Yes

\paragraph{Dependencies}
Object destructuring supports the computed property notation of extended object literals (as discussed in Appendix \ref{object-literals}):

\begin{lstlisting}
var qux = "key";
var { [qux] : a } = obj;
\end{lstlisting}

Destructuring thus has a dependency on extended object literal notation for this feature to work properly.

\paragraph{Decomposable}
This language features consists of two types of destructuring, object destructurings, and array destructurings. These two different features can be transformed separately.

\section{Extended object literals} \label{object-literals}
Literal object notation receives three new features in the ECMAScript 6 standard~\cite[12.2.5]{SpecJS}. Shorthand property notation, shorthand method notation, and computed property names.

\paragraph{Category}
The extended object literals are syntactic sugar. There are direct rules to eliminate the introduced concepts to core concepts of the ECMAScript 5 language. (e.g. shorthand property notation translate to \textit{property: value} notation)

\paragraph{Introducing bindings}
One run-time function is used by the transformation suite, to set properties with computed property names. This method is a wrapper for \textit{Object.defineProperty}\footnotemark method. By using this run-time function instead of \textit{Object.defineProperty} directly we avoid unnecessary "noise".
\footnotetext{For more information on this function see the \href{https://developer.mozilla.org/en/docs/Web/JavaScript/Reference/Global_Objects/Object/defineProperty}{Mozilla Developer Network documentation}}

\paragraph{Decomposable}
All three extensions of the object literal can be transformed separately.

\section{For of loop} \label{for-of}
The ECMAScript 6 standard introduces iterators, and a shorthand \textit{for-loop} notation to loop over these iterators. This loop is called the \textit{for of} loop~\cite[13.6.4]{SpecJS}. Previous versions of the ECMAScript standard had default for loops, and \textit{for in} loops (which iterate over all enumerable properties of an object and those inherited from its constructor's prototype).

\paragraph{Category}
This construct can be eliminated by transformation to a \textit{for-loop} using an iterator variable, and selecting the bound variable for the current index from the array using this iterator variable.

\paragraph{Constraints on sub-terms}
No constraints any $<Statement>$ can be used within the body of a for-of loop.

\begin{lstlisting}{language=rascal}
syntax Statement
	= "for" "(" Declarator ForBinding "of" Expression ")" Statement;
\end{lstlisting}

\paragraph{Dependencies}
The for-of loop binding can be declared with the let declarator (see Appendix \ref{let-const}).

\begin{lstlisting}
for( let x of arr ) <Statement>
\end{lstlisting}

Destructuring assignment can be used as a for-of loop binding (see Appendix \ref{destructuring}):

\begin{lstlisting}
for( [a,b] of arr ) <Statement>
\end{lstlisting}

\section{Spread operator}
ECMAScript 6 introduces a new unary operator named spread\cite[12.3.6.1]{SpecJS}. This operator is used to expand an expression in places where multiple arguments (i.e. function calls) or multiple elements (i.e. array literals) are expected.

\paragraph{Category}
This unary operator can be expressed using core concepts of the ECMAScript 5 language. In case the operator appears in a place where multiple arguments are expected, a call of member function \textit{apply}\footnotemark on our function can be used to supply arguments as an array. In case it appears in a place where multiple elements are expected the prototype function \textit{concat} of arrays can be used to interleave the supplied argument to the operator with the rest of the array.

\footnotetext{For more information see \href{https://developer.mozilla.org/en-US/docs/Web/JavaScript/Reference/Global_Objects/Function/apply}{Mozilla Developer Network documentation}}

Every function in JavaScript is an object, Function.prototype has a method called \textit{apply} which can be used to invoke a function, the first argument of \textit{apply} is the object to bind to \textit{this} during function execution (see Section \ref{javascript-scoping}). And the second argument is an array whose values are passed as arguments to the function:

\begin{minipage}{0.45\textwidth}
\begin{lstlisting}
f(...arr);
\end{lstlisting}
\end{minipage}
\hfill
\begin{minipage}{0.45\textwidth}
\begin{lstlisting}
f.apply(f, arr);
\end{lstlisting}
\end{minipage}

\begin{minipage}{0.45\textwidth}
\begin{lstlisting}
[x, ...arr];
\end{lstlisting}
\end{minipage}
\hfill
\begin{minipage}{0.45\textwidth}
\begin{lstlisting}
[x].concat(arr);
\end{lstlisting}
\end{minipage}

\paragraph{Introducing bindings} \label{spread-intoducing-bindings}
As explained above a call to \textit{apply} on a Function's prototype requires the callee to set the object that is bound to \textit{this} during execution of the function. When a function is called as a property of an object, this object should be bound to \textit{this} during the function's execution (and not the function as shown above). The following example illustrates this behavior:

\begin{lstlisting}
a.f(...args);
\end{lstlisting}
\begin{lstlisting}[caption={apply function with correct $this$ scope}]
a.f.apply(a,args);
\end{lstlisting}

If however the object (\textit{a} in this case) is itself a property of a different object we need to bind this value so it can be used as the first argument of \textit{apply}. The following example illustrates this special case:

\begin{minipage}{0.45\textwidth}
\begin{lstlisting}
a.b.f(...args);
\end{lstlisting}
\end{minipage}
\hfill
\begin{minipage}{0.45\textwidth}
\begin{lstlisting}
var _b$a;
(_b$a = b.a).f.apply(_b$a, args);
\end{lstlisting}
\end{minipage}

If there exists a function call of function $f$ on object $a$ where spread operator is used in arguments list. This function call has to be transformed to a call to function $apply$ on function $f$ on object $a$. Where the first parameter of $apply$ is $a$ and second is the argument of spread. The sub-terms of function calls on objects with spread operators thus have to be analyzed to determine the call scope of the called function.

This introduces a new binding (\textit{\_b\$a}).

\paragraph{Depending on bindings}
For correct specification compliance the argument of the spread operator has to be transformed to a consumable array. This behavior can be simulated through a run-time function.

\paragraph{Analysis of sub-terms} \label{spread-analysis-sub-terms}
When the pattern \lstinline$<Expression e>(<{ArgExpression ","}* args>)$ is matched \lstinline$e$ has to be analyzed to detect whether it is a function, or object property function. (See Appendix \ref{spread-intoducing-bindings})

\paragraph{Decomposable}
The operator can be used in two places (places of multiple arguments, and multiple elements), which can be transformed separately.

\section{Default parameter}
The ECMAScript 6 spec defines a way to give parameters default values~\cite[9.2.12]{SpecJS}. These values are used if the caller does not supply any value on the position of this argument. Any default value is evaluated in the scope of the function (i.e. $this$ will resolve to the run-time context of the function the default parameter value is defined on).

\paragraph{Category}
This language feature can be eliminated by transformation to a core concept of the ECMAScript 5 language. By binding the variable in the function body to either the default value (in case the parameter equals $undefined$) or to the value supplied by the parameter.

\paragraph{Introducing bindings}
This transformation does not introduce any bindings because the bindings already exist as formal parameters.

\section{Rest parameter}
The \textit{BindingRestElement} defines a special parameter which binds to the remainder of the arguments supplied by the caller of the function.\cite[14.1]{SpecJS}

\paragraph{Category}
The rest element can be retrieved in the function's body through the use of a \textit{for} loop. All arguments of a function are stored inside the implicit variable \textit{arguments}\footnotemark (\textit{arguments} is an array like object). The rest argument is a slice of the \textit{arguments} object, where the start index is the number of parameters (minus the rest parameter) of the function, and until the size of \textit{arguments}. This slice can be obtained by a \textit{for} loop.

Rest parameter is eliminated to core ECMAScript 5 language construct, this feature is syntactic sugar.

\paragraph{Syntactically type preserving}
$<Function>$ to $<Function>$

\section{Template Literal}
Standard string literals in JavaScript have some limitations. The ECMAScript 6 specification introduces a template literal to overcome some of these limitations~\cite[12.2.8]{SpecJS}. Template literals are delimited by the ` quotation mark, can span multiple lines and be interpolated by expressions: \lstinline{$<Expression>}

\paragraph{Category}
Template strings are syntactic sugar for Strings concatenated with expressions and new-line characters.

\paragraph{Syntactically type preserving} \mbox{}\\
\lstinline$<Expression <TemplateLiteral>>$ to \lstinline$<Expression <StringLiteral>>$

\section{Tail call optimization} \label{tail-call-optimization}
JavaScript knows no optimization for recursive calls in the tail of a function, this results in the stack increasing with each new recursive call overflowing the stack as a result.  The ECMAScript 6 specification introduces tail call optimization to overcome this problem~\cite[14.6]{SpecJS}

\paragraph{Category}
Optimization, this transformation improves the space performance of a program.

\paragraph{Abstraction level}
Tail call optimization is a semantic based transformation. To identify if the expression of a return statement is a recursive call to the current function we need name binding information.

\paragraph{Program modification}
This extension is purely semantic, no new syntax is introduced, thus a program modification and not extension. To transform this extension to work in older JavaScript interpreters the tail call has to be removed and replaced by a \lstinline$while$ loop, to prevent the call stack from overflowing.

\begin{lstlisting}[caption={Function with tail recursion},label={lst:tail-call}]
function fib(n,nn,res) {
	if( nn > 100 ) return res;
	return fib(nn,n + nn,res + [n,nn]);
}
\end{lstlisting}

\begin{lstlisting}[caption={Semantically identical function, without tail recursion},label={lst:transformed-tail-call}]
function fib(arg0,arg1,arg2) {
	while (true) {
		var n = arg0, nn = arg1, res = arg2;

		if(nn > 100) return res;
		arg0 = nn;
		arg1 = n + nn;
		arg2 = res + [n,nn];
	}
}
\end{lstlisting}

Figure \ref{lst:transformed-tail-call} illustrates how a function with tail recursion can be transformed to a function with an infinite loop. Not filling the call stack with new return positions and thus executing correctly even when our upper limit (100) is increased.

\section{Generators} \label{generators}
The ECMAScript 6 standard introduces a new function type, generators~\cite[14.4]{SpecJS}. These are functions that can be exited and later re-entered, when leaving the function through $yielding$ their context (i.e. variable bindings) is saved and restored upon re-entering the function. A generator function is declared through \lstinline$function*$ and returns a Generator object (which inherits from $Iterator$)

\paragraph{Category}
A generator can be implemented as a state machine function wrapped in a little run-time returning a Generator object. It does categorize as syntactic sugar, however the transformation is complex (because of identification of control flow of body statement, see Analysis of sub-terms).

\paragraph{Scope}
The transformation can be managed local-to-local.

\paragraph{Syntactically type preserving} \mbox{}\\
\lstinline$<Statement>$ to \lstinline$<Statement>$ \\
\lstinline$<Expression>$ to \lstinline$<Expression>$

\paragraph{Analysis of sub-terms}
The control flow of the functions body has to be identified. The control flow graph can be interpreted as a state machine with several leaps, entry points, and exit points that result from the use of different control-flow constructs (e.g. $yielding$, $returning$, $breaking$, etc.). After transformation this state machine can be represented in a \lstinline$switch$ statement. All bindings of the function body have to be hoisted outside of this state machine, and only to be assigned through an assignment expression in the state machine.

\paragraph{Preconditions}
The \lstinline$<GeneratorBody>$ has to be transformed before transformation of the Generator function starts.

\paragraph{Depending on bindings}
The state machine function is wrapped in a run-time function to initialize the Generator object returned from a generator function. The run-time makes sure the state-machine has correct scoping bindings on each entry\footnote{\url{http://github.com/facebook/regenerator/blob/master/runtime.js}}

\section{Let and Const declarators} \label{let-const}
JavaScript's scoping happens according to the lexical scope of functions. Variable declarations are accessible in the entire scope of the executing function, this means no block scoping. The \lstinline$let$ and \lstinline$const$ declarator of the ECMAScript 6 specification change this. Variables declared with either one of these is lexically scoped to its block and not its executing function. The variables are not hoisted to the top of their block, they can only be used after they are declared. Code before the declaration of a \lstinline$let$ variable is called the temporal dead-zone. \lstinline$const$ declaration are identical to \lstinline$let$ declarations with the exception that reassignment to a \lstinline$const$ variable results in a syntax error. This aids developers in reasoning about the (re)binding of their const variables. Do not confuse this with immutability, members of a \lstinline$const$ declared object can still be altered by code with access to the binding.

\paragraph{Category}
It is possible to eliminate \lstinline$let$ declarations without losing correct block scoping of bindings. We can illustrate this with the help of a simple example. 0 is bound to \lstinline$x$, now we bind 1 to \lstinline$x$ in a nested block. Outside of this block 1 is still bound to $x$.

\begin{lstlisting}
let x = 1;

{
	let x = 0;
	x == 0; // true
}

x == 0; // false
\end{lstlisting}

This behavior can be emulated in ECMAScript 6 with the use of immediately invoking function expressions (or IIFE), the function expression introduces a new function scope. And the rebinding of x is not `visible` outside of the new IIFE.

\begin{lstlisting}
var x = 1;

(function() {
	var x = 0;
	x == 0; // true
})();

x == 0; // false
\end{lstlisting}

Thus $let$ declarations can be desugared if every block is transformed to a \lstinline$closure$ with the help of IIFE's. This does however imply a serious performance overhead on the transformed program (instead of normal block execution the interpreter has to create a closure for every block). This also breaks the standard control-flow of our program when the block uses any control-flow statement keywords (e.g. \lstinline$return$).

A second way to emulate block binding in ECMAscript 6 is with the help of a \textit{try catch} statement, the introduced binding can be thrown in the try block and caught by the catch block. Making the binding available only to the remaining statements enclosed in the catch block. The issue with this solution is performance overhead imposed by \textit{try catch} blocks, because these are not optimized by the interpreter of the JavaScript code.

There is another way to transform \lstinline$let$ declarations, for this implementation we need full name analysis of a program. The previous example can also be transformed by renaming the rebinding of \lstinline$x$:

\begin{lstlisting}
var x = 1;

{
	var x_1 = 0;
	x_1 == 0; // true
}

x == 0; // false
\end{lstlisting}

The analysis has access to both the binding graph of the program with function-scoping and \lstinline$var$ declarations, and the binding graph with block scoping and \lstinline$let$ declarations. After the creation of these graphs an illegal reference can be defined as a reference that does appear in the function-scope binding graph, but does not appear in the block-scope binding graph. In the case of our example this means that \lstinline$x$ references 0 bound to \lstinline$x$ in block (at line 4) when analyzing function-scope bindings. But \lstinline$x$ does not reference the same binding when analyzing the block-scope binding graph. \footnote{More examples: \url{http://raganwald.com/2015/05/30/de-stijl.html}}

\paragraph{Scope}
The transformation needs to analyse the entire program (global) and make transformations to its entirety (global). Thus the scope of the transformation is global-to-global.

\paragraph{Compositional}
The transformation of \lstinline$let$ (or \lstinline$const$) bindings is not compositional because bindings of these types can not be captured by names in sibling scopes

\begin{lstlisting}
{
	let x = 0;
}
{
	let x = 1;
}
\end{lstlisting}

Once these \lstinline$let$ bindings are transformed to \lstinline$var$ bindings without renaming they would be declared in the same lexical scope. Whereas they are not at the moment. For a transformation to \lstinline$var$ bindings to work one of these bindings and its uses has to be renamed.

\paragraph{Introducing bindings}
The transformation introduces new bindings, it converts \lstinline$let$ and \lstinline$const$ bindings to \lstinline$var$ bindings, and renames these bindings (and their references) where necessary.

\paragraph{Preconditions}
Other transformations need to be performed before the let const extension is transformed. To prevent this extension to have dependencies on other extensions. For example the for-of extension has a dependency on the let-const extension:

\begin{lstlisting}
for( let x of arr ) <Statement>
\end{lstlisting}

If the for-of loop is transformed to a normal loop let-const needs not to be aware of the for-of loop as an extension, but just of the ECMAScript 5 manner to declare variables inside loops.

\paragraph{Dividable}
Variables declared through let and const can be transformed to ECMAScript 5 code separately.
