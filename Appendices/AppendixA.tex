% Appendix A

\chapter{Artifact Description} % Main appendix title

\label{AppendixA} % For referencing this appendix elsewhere, use \ref{AppendixA}

\lhead{Appendix A. \emph{Artifact Description}} % This is for the header on each page - perhaps a shortened title

\paragraph{Summary}
We provide implementation of ECMAScript 6 language features syntax, and transformation rules in the Rascal meta-programming language (\url{http://rascal-mpl.org}). We use rascal's built-in support for syntax definition and parsing.

\paragraph{Content}
The code for language extensions is stored in \textit{src/extensions} here we will discuss implementation status for each language feature:

\paragraph{Function extensions}\mbox{}\\
\textbf{Arrow functions} (\textit{src/extensions/arrow}) \newline
Full functionality\textsuperscript{*} (i.e. basic functionality and correct \textit{this},\textit{arguments} binding)

\textbf{Class declaration} (\textit{src/extensions/classes}) \newline
Full functionality.\textsuperscript{*} (i.e. basic support, support for extension, getter/setter/static methods, methods are non-enumerable)

\textbf{Super} (\textit{src/extensions/classes}) \newline
Full functionality.\textsuperscript{*}

\textbf{Generators} (\textit{src/extensions/generators}) \newline
No support. Due to the time constraint nature of the project this feature has not been implemented. 

\paragraph{Syntax extensions}\mbox{}\\
\textbf{Destructuring} (\textit{src/extensions/destructuring}) \newline
Near full functionality. There is no support for unbound match:
\begin{lstlisting}
var [,b] = [1,2]; // b == 2
\end{lstlisting}
This feature is not implemented because syntax definition for empty elements in a comma separated list fairly complex. Making the resulting concrete syntax even more complex. To avoid unneeded complexity we decided to refrain from implementation of this feature.

\textbf{For of loops} (\textit{src/extensions/forof}) \newline
Full functionality.

\textbf{Binary \& Octal literals} (\textit{src/extensions/literal}) \newline
Full functionality.

\textbf{Object literal} (\textit{src/extensions/object}) \newline
Full functionality.

\textbf{Rest Parameters} (\textit{src/extensions/parameters}) \newline
Near full functionality. (i.e. no support for arguments object interaction outside of strict mode).

\textbf{Default} (\textit{src/extensions/parameters}) \newline
Basic functionality.

\textbf{Spread operator} (\textit{src/extensions/spread}) \newline
Near full functionality. (i.e. no support for spread operator with generators, because we do not implement generators)

\textbf{Template Literals} (\textit{src/extensions/template}) 
Full functionality (i.e. tagged template literals and normal template literals)

\paragraph{Binding extensions}\mbox{}\\
\textbf{Let} (\textit{src/extensions/letconst}) \newline
Full functionality. (i.e. support for renaming of function scoped clashes of block scoped names, no reference possible before definition, and possibility to throw run-time errors on redeclaration)

\textbf{Const} (\textit{src/extensions/letconst}) \newline
Full functionality. (i.e. same as let)

\textbf{Modules} (\textit{src/extensions/modules}) 
No support.

\textsuperscript{*} These features make use of the ES6 \textit{new.target} property. This newly introduced property is an implicit parameters. Implicit parameters are those parameters that are set during every function executed but not explicitly passed by the caller (e.g. \textit{arguments}). It is used to refer to the constructor function when a function is invoked with the use of the \textit{new} keyword.
Our implementation does not support the \textit{new.target} property. It is possible to parse a new.target references but we have not implemented any desugaring code to transform the reference to ES5 code. The dynamic behavior of \textit{new.target} makes the transformation complex to perform during a static (i.e. non-runtime) phase. A transformation would be global-to-global (see:\ref{AppendixB}) (every function call has to be analyzed in the entire program).

The compatibility tests as discussed in \ref{} are stored in \textit{input/compatibility} and can be invoked from the Rascal module at \textit{src/test/Compatibility.rsc}

Our implementation of the \textit{name-fix}\citep{Erdweg2014a} algorithm resides in \textit{src/core/resolver}. 

\paragraph{Getting the artifact}
The latest version of the artifact is available at \url{https://github.com/matthisk/rascal-sweetjs} as a git repository. The repository page includes information on how to run the artifact inside the Eclipse IDE\footnote{\url{http://eclipse.org}}.