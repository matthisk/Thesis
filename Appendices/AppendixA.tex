% Appendix A

\chapter{Artifact Description} % Main appendix title

\label{AppendixA} % For referencing this appendix elsewhere, use \ref{AppendixA}

\lhead{Appendix A. \emph{Artifact Description}} % This is for the header on each page - perhaps a shortened title

\paragraph{Summary}
We provide implementation of ECMAScript 6 language features syntax, and transformation rules in the Rascal meta-programming language (\url{http://rascal-mpl.org}). We use rascal's built-in support for syntax definition and parsing.

\paragraph{Content}
The code for language extensions is stored in \textit{src/extensions} here we will discuss implementation status for each language feature:

\paragraph{Core}
\textbf{Syntax} (\textit{src/core/Syntax.rsc}) \newline
The core JavaScript syntax definition. There are two differences with the specification, semicolons are required (the ECMAScript 5 defines a functionality called automatic semicolon insertion or asi, with asi semicolons are not required and can be inferred by the parser). Because we use the Rascal syntax definition (and not a custom parser) we do not support asi. For the same reason there is no support for comma expression, the introduction of this expression causes ambiguities in the grammar.

\paragraph{Function extensions}\mbox{}\\
\textbf{Arrow functions} (\textit{src/extensions/arrow}) \newline
Full functionality\textsuperscript{*} (i.e. basic functionality and correct \textit{this},\textit{arguments} binding)

\textbf{Class declaration} (\textit{src/extensions/classes}) \newline
Full functionality.\textsuperscript{*} (i.e. basic support, support for extension, getter/setter/static methods, methods are non-enumerable)

\textbf{Super} (\textit{src/extensions/classes}) \newline
Full functionality.\textsuperscript{*}

\textbf{Generators} (\textit{src/extensions/generators}) \newline
No support. Due to the time constraint nature of the project this feature has not been implemented. 

\paragraph{Syntax extensions}\mbox{}\\
\textbf{Destructuring} (\textit{src/extensions/destructuring}) \newline
Near full functionality. There is no support for unbound match:
\begin{lstlisting}
var [,b] = [1,2]; // b == 2
\end{lstlisting}
This feature is not implemented because syntax definition for empty elements in a comma separated list is complex. Resulting in very unclear/cluttered/complicated transformation rules/code. To avoid this complexity we decided to refrain from implementation of this feature.

\textbf{For of loops} (\textit{src/extensions/forof}) \newline
Full functionality.

\textbf{Binary \& Octal literals} (\textit{src/extensions/literal}) \newline
Full functionality.

\textbf{Object literal} (\textit{src/extensions/object}) \newline
Full functionality.

\textbf{Rest Parameters} (\textit{src/extensions/parameters}) \newline
Near full functionality. (i.e. no support for arguments object interaction outside of strict mode).

\textbf{Default} (\textit{src/extensions/parameters}) \newline
Basic functionality.

\textbf{Spread operator} (\textit{src/extensions/spread}) \newline
Near full functionality. (i.e. no support for spread operator with generators, because we do not implement generators)

\textbf{Template Literals} (\textit{src/extensions/template}) 
Full functionality (i.e. tagged template literals and normal template literals)

\paragraph{Binding extensions}\mbox{}\\
\textbf{Let} (\textit{src/extensions/letconst}) \newline
Full functionality. (i.e. support for renaming of function scoped clashes of block scoped names, no reference possible before definition, and possibility to throw run-time errors on redeclaration)

\textbf{Const} (\textit{src/extensions/letconst}) \newline
Full functionality. (i.e. same as let)

\textbf{Modules} (\textit{src/extensions/modules}) 
No support.

\textsuperscript{*}These features make use of the ES6 \textit{new.target} property. This newly introduced property is an implicit parameters. Implicit parameters are those parameters that are set during every function execution but not explicitly passed as an argument by the caller. It is used to refer to the constructor function when a function is invoked with the use of the \textit{new} keyword.
Our implementation does not support the \textit{new.target} property. It is possible to parse a new.target references but we have not implemented any transformation code to transform the property reference to ES5 code. The dynamic behavior\footnotemark of \textit{new.target} makes the transformation complex to perform during a static (i.e. non-runtime) phase. To transform \textit{new.target} a global analysis of the program would be needed and a global transformation (each function needs to be analyzed and each function call), thus a global-to-global transformation (see Section \ref{taxonomy}).

\footnotetext{\textit{new.target} refers to the constructor function called with the new keyword, even inside the paren constructor function invoked through \textit{super}. It is undefined in functions that are called without new keyword. The behavior can be overridden with the use of \textit{Reflect.construct}}

The compatibility tests as discussed in \ref{} are stored in \textit{input/compatibility} and can be invoked from the Rascal module at \textit{src/test/Compatibility.rsc}

Our implementation of the \textit{name-fix}\citep{Erdweg2014a} algorithm resides in \textit{src/core/resolver}. 

\paragraph{Getting the artifact}
The latest version of the artifact is available at \url{https://github.com/matthisk/rascal-sweetjs} as a git repository. The repository page includes information on how to run the artifact inside the Eclipse IDE\footnote{\url{http://eclipse.org}}.