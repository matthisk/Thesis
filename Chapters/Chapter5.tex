% Chapter Template

\chapter{Conclusion} % Main chapter title

\label{Chapter5} % Change X to a consecutive number; for referencing this chapter elsewhere, use \ref{ChapterX}

\lhead{Chapter 5. \emph{Conclusion}} % Change X to a consecutive number; this is for the header on each page - perhaps a shortened title

\section{Related Work}
Various research has been performed into the creation of tools to aid extension of programming languages and IDE's by meta-programmers. Eelco Visser~\cite{Visser20024} presents a system for language extension, in this system grammar's are defined using Syntax Definition Formalism and program transformations are implemented using the Stratego language. Visser ignores the cross-cutting concerns of program transformations (e.g. variable capture). Erdweg et. al.~\cite{Erdweg} present a system for extensible languages named Sugar*, which is agnostic from base language. The system can extend- syntax, editor support, and static analysis. In contrast to the system presented in this thesis, Sugar* relies on information from a compiler of the base language to operate correctly.  
All of the work discussed above present minimal examples of actual language extensions implemented using their tools for extensible languages. The specific extension of the JavaScript programming language is studied by Disney et. al. ~\cite{Disney2014}, here JavaScript is extended through the use of syntax macros instead of separate language extensions as presented in this thesis.

In related work of our taxonomy of language extensions (section \ref{taxonomy}) we can identify several categorizations of program transformations. For example the Irvine program transformation catalog~\cite{Standish1976a} presents a categorization for program transformations for lower-level procedural programming languages (e.g. the C programming language). Visser~\cite{Visser2001} presents a taxonomy for program transformations, which we partly reuse in our taxonomy of language extensions. 

Other aspects of program transformations have also received attention in the past. In this thesis we reuse the \textit{name-fix} algorithm presented by Erdweg et. al.~\cite{Erdweg2014}. Transformation hygiene has received a lot of attention in the domain of language macros~\cite{Kohlbecker1986,Herman2010a,Disney2014}

\section{Conclusions}
