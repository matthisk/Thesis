% Chapter Template

\chapter{Conclusion} % Main chapter title

\label{Chapter5} 

\lhead{Chapter 5. \emph{Conclusion}}

\section{Related Work}
Many researchers have tried to make programming languages extensible, be it through language extensions, compiler extensions, or syntax macros. Often their research also tries to extend the tools with which programmers create their programs. Eelco Visser~\cite{Visser20024} presents a system for language extension through concrete syntax. In this system grammar's are defined using Syntax Definition Formalism~\cite{Heering1989} and program transformations are implemented using the Stratego~\cite{Visser2001a} transformation language. The system is modular and composable this makes it possible to combine different language extensions. The main focus of Visser's research is in the area of defining and parsing additional syntax and embedding programming languages. Visser ignores the cross-cutting concerns of program transformations (e.g. variable capture). Another system based on SDF and Stratego is presented by Erdweg et. al.~\cite{Erdweg} called Sugar*. This is a system for language extensibility agnostic from base language, the system can extend- syntax, editor support, and static analysis. In contrast to the system presented in this thesis, Sugar* relies on information from a compiler of the base language to operate correctly. Sugar* is based on SugarJ~\cite{Erdweg2011} library based language extension tool for the Java programming language.  All of the work discussed above focusses mainly on the embedding of programming languages inside a host language (e.g. using XML inside Java programs) in their case studies. Aside from these examples no \textit{real-world} language extensions are demonstrated in their work. The extension of the JavaScript programming language is studied by Disney et. al. ~\cite{Disney2014}, here JavaScript is extended through the use of syntax macros instead of separate language extensions as presented in this thesis.

In related work of our taxonomy of language extensions (section \ref{taxonomy}) we can identify several categorizations of program transformations. For example the Irvine program transformation catalog~\cite{Standish1976a} presents a categorization for program transformations for lower-level procedural programming languages (e.g. the C programming language). Visser~\cite{Visser2001} presents a taxonomy for program transformations, which we partly reuse in our taxonomy of language extensions. 

Other aspects of program transformations have also received attention in the past. In this thesis we reuse the \textit{name-fix} algorithm presented by Erdweg et. al.~\cite{Erdweg2014}. Transformation hygiene has received a lot of attention in the domain of syntax macros~\cite{Kohlbecker1986,Herman2010a,Disney2014}. 

\section{Conclusions}
