% !TEX root = ../main.tex
% Chapter Template

\chapter{Conclusion} % Main chapter title

\label{Chapter7} 

\lhead{Chapter 7. \emph{Conclusion}}

In this thesis we have made two contributions. First, a taxonomy for language extensions. Second, we have tried to show the \textit{effectiveness} of a language workbench for the task of implementing language extensions. We implemented ES6 language extensions in \projectname as an experiment to guide evaluation of the language workbench. And evaluated the implementation against other transpilers capable of transforming ES6 to ES5.

The taxonomy can be used to create a categorization of language extensions and estimate the work needed to implement transformation code for a specific extension. The categorization created for the ES6 language extensions proved valuable in predicting problems with the implementation of language extensions (e.g. can variable capture arise during transformation of the language extension).

The language workbench has been a valuable tool for the implementation of language extensions. However it is inconclusive if it really improves on all aspects as compared to traditional implementations. The issue of portability does impose a large restriction on our implementation, advantages of the language workbench are integration with developer tools, expressive transformation code using concrete syntax patterns, and modular language extensions. If a fixed set of language extensions has to be implemented to be distributed to a large group of developers, it is arguably better to take a traditional approach, because parser code will not change much overtime and portability issues of the language workbench can be avoided. If however the set of language extensions is not fixed and the language extensions are used by a small team the language workbench can present a good option for implementation of the language extensions.