% !TEX root = ../main.tex
% Chapter Template

\chapter{Taxonomy} % Main chapter title

\label{Chapter4}

\lhead{Chapter 4. \emph{Taxonomy}} \label{taxonomy}

Every language extension has several properties which can be identified and categorized along certain dimensions. In this chapter we present a taxonomy for language extensions, such a taxonomy can be used for several purposes. The taxonomy helps identifying the following aspects of language extensions, (1) implementation intricacies of language extensions, (2) Implementation type (e.g. through the use of a program transformation), (3) What is the relation between language extensions. In appendix \ref{AppendixB} we categorize ES6 language features according to this taxonomy.

\section{Structure}

The several dimensions of the taxonomy give insight in the answers to three questions specific to a language extension.

\textit{How can the language extension be implemented?}\\
There are multiple ways to implement a language extension (e.g. a program transformation or a compiler extension). Sometimes one solution is preferable over another, the following dimensions help in selecting an implementation type for a language extensions:

\begin{itemize}
	\item Category
	\item Abstraction level
	\item Extension or modification
	\item Dependencies
	\item Decomposable
\end{itemize}

\textit{What information does transformation of a language extension require?}\\
Transformations rely on information in the source program to be able to perform transformations. Some require more information than others, this additional information required can make implementation of the language extension more complex. The following dimensions help in identifying the amount of context needed for a language extension to be transformed:

\begin{itemize}
	\setlength\itemsep{0.4em}
	\item Category
	\item Compositionality
	\item Analysis of subterms
	\item Scope
\end{itemize}

\textit{What are guarantees can we give of the target program produced by a language extension?}\\
Transformation of a language extension produces a target program different from the source program. The following dimensions help identifying to what extend target and source program differ from each other, giving insight into the inner working of a transformation:

\begin{itemize}
	\item Syntactically type preserving
	\item Introduction of bindings
	\item Depending on bindings
	\item Backwards compatible
\end{itemize}

\section{Dimensions}
The following dimensions are identified and used to categorize every language extension. Each of the following paragraphs investigates a relevant question to language extensions and provides insight into answering these question with the use of examples.

\paragraph{Category}
One of the rephrasing categories defined by Eelco Visser~\cite{Visser2001}, rephrasings are program transformations where source and target program language are the same. Here we discuss each category.

\textbf{Normalization}
The reduction of a source program to a target program in a sub-language of the source program language.

\begin{description}[font=\normalfont,align=right,leftmargin=!, labelwidth=\widthof{{\bfseries Specialization}},labelsep=2em]
	\item[Desugaring] a language construct (called syntactic sugar) is reduced to a core language.
	\item[Simplification] this is a more generic form of normalization in which parts of the program are transformed to a standard form with the goal of simplifying the program without changing the semantics.
	\item[Weaving] this transformation injects functionality in a source program, without modifying the code. It is used in aspect-oriented programming, where cross-cutting concerns are separated from the main code and later 'weaved' with the main program through an aspect weaver.)                                                                                                                                                                              
\end{description}

\textbf{Optimization}
These transformations help improve the run-time and/or space performance of a program                                                                                        

\begin{description}[font=\normalfont,align=right,leftmargin=!, labelwidth=\widthof{{\bfseries Specialization}},labelsep=2em]
	\item[Specialization] Code specialization deals with code that runs with a fixed set of parameters. When it is known that some function will run with some parameters fixed the function can be optimized for these values before run-time. (e.g. compiling a regular expression before execution).                                                                                                                                                                              
	\item[Inlining] Transform code to inline a certain (standard) function within your function body instead of calling the function from the (standard) library. This produces a slight performance increase because we avoid an additional function call. (this technique is more common in lower level programming languages e.g. C or C++)                                                                                                                                                                             
	\item[Fusion] Fusion merges two bodies of loops (or recursive code) that do not share references and loop over the same range, with the goal to reduce running time of the program.                                                                                                                                                                              
\end{description}

\textbf{Other}

\begin{description}[font=\normalfont,align=right,leftmargin=!, labelwidth=\widthof{{\bfseries Specialization}},labelsep=2em]
	\item[Refactoring] is a disciplined technique for restructuring an existing body of code, alteringits internal structure without changing its external behaviour.\footnotemark   
	\item[Obfuscation] is a transformation that makes output code less (human) readable, while not changing any of the semantics.
	\item[Renovation] is a special form of refactoring, to repair an error or to bring it up to date with respect to changed requirements~\cite{Visser2001}
\end{description}
\footnotetext{\url{http://www.refactoring.com}}

\paragraph{Abstraction level}
Program transformations can be categorized by their abstraction level. There are four levels of abstraction (similar to those of macro expansions~\cite{Weise1993}), character-, token-, syntax-, or semantic-based. Character and token based transformations work on a program in textual representation. Syntactical transformations work on a program in its parsed representation (either as an AST or as a parse tree, see section \ref{program-representation}). In addition to the syntactic representation semantic transformations also have access to the static semantics of the input program (e.g. variable binding).

\paragraph{Extension or Modification}
Rephrasings try to say the same thing (i.e. no change in semantics) but using different words\cite{Visser2001}. Sometimes these different words are an extension on the core language, in this case we call the transformation a \textit{program extension}. In other cases the transformation uses only the words available in the core language, then we call the transformation a \textit{program modification}. Transformations that fall in the \textit{optimization} category (see table \ref{table-rephrasing-categories}) are program modifications. An example is tail call optimization in which a recursive function call in the \textit{return} statement is reduced to a loop to avoid a call-stack overflow error (see appendix \ref{tail-call-optimization}).

\paragraph{Scope}
Program transformations performed on the abstraction level of context-free syntax (or semantics) receive the parse tree of the source program as their input. A transformation searches the parse tree for a specific type of node, the type of node to match on is defined by the transformation and can be any syntactical type defined in the source program's grammar. The node matched by a transformation and whether or not information from outside this node's scope is used during transformation determine the scope of a program transformation, there are four different scopes:

\begin{enumerate}
	\item When a program transformation matches on a sub-tree of the parse-tree and only transforms this matched sub-tree it is a \textit{local-to-local} transformation. 
	\item If the transformation needs information outside the context of the matched sub-tree, but only transforms the matched sub-tree it is \textit{global-to-local}. 
	\item When a transformation has no additional context from its local sub-tree but does alter the entire parse-tree it is called \textit{local-to-global}. 
	\item If the transformation transforms the input program in its entirety it is \textit{global-to-global}.
\end{enumerate}

\paragraph{Syntactically type preserving}
Program transformations performed on syntax elements can preserve the syntactical type of their input element or alter it. Two main syntactical types in JavaScript are Statement and Expression (see section \ref{javascript-syntax}). If a transformation matches on a Expression node but returns a Statement it is not syntactically type preserving.

\paragraph{Introduction of bindings}
Transformations that do not introduce bindings are guaranteed to be hygienic (see section \ref{hygiene}), where binding introducing transformations can cause variable capture from synthesized bindings to source bindings.

\paragraph{Depending on bindings (i.e. runtime system)}
Will the target program produced by the transformation depend on context not introduced by the transformation (e.g. global variables, external libraries).

\paragraph{Compositional}
When a program transformation does not alter the containing context of the matched parse-tree node, it is said to be compositional. The main concern of compositionality of program transformations is if the transformation can be reversed or not.

\paragraph{Preconditions}

...

What are the preconditions that have to be met before execution of a transformation rule, to ensure validity of the transformation (e.g. all sub-terms have to be analyzed and transformed)

\paragraph{Restrictions on sub-terms}
Does the language extensions impose restrictions on the terms used inside of the language extension's non-terminals. For example we can only use identifiers in the sub-terms of our \lstinline$swap$ language extension (see section \ref{hygiene}).

\paragraph{Analysis of sub-terms}
Are the non-terminals of our language extension analyzed by the transformation rule. This is related to compositionality but differs because non-compositional transformations analyze \textit{and transform} sub-terms.

\paragraph{Dependency on other extensions}
Can the language extensions be performed stand-alone or is there a dependency on one of the other extensions.

\paragraph{Backwards compatible}
Is all valid base language code also valid inside an environment where the base language is extended with this language extension. For example the \textit{let-const} (see appendix \ref{let-const}) language extension introduces a new keyword \lstinline$let$, because of this code written in the base language could break when parsed with the \textit{let-const} language extension enabled. For instance \lstinline$let[x] = 10;$ would be correct ES5 JavaScript code in ES6 this is however not allowed because \lstinline$let$ is a reserved keyword.

\paragraph{Decomposable}
Is it possible to identify smaller transformation rules inside this language extension, that can be performed independently from one another.

\begin{landscape}		
\centering

\begin{table}[h]
\caption{ES6 features transformation dimensions}
\label{full-table}
\begin{tabular}{rcccccc}
\hline
& {\bf Arrow Functions} & {\bf Classes} & {\bf Destructuring} & {\bf Object literals} & {\bf For of loop} & {\bf Spread operator} \\ \hline
{\bf Category} & D. & D. & D. & D. & D. & D. \\
{\bf Abstraction level} & CfS. & CfS. & CfS. & CfS. & CfS. & CfS. \\
{\bf Scope} & G2L & L2L & L2L & L2L & L2L & L2L \\
{\bf Extension or Modification} & E. & E. & E. & E. & E. & E. \\
{\bf Syntactically type preserving} & $\bullet$ & $\bullet$ & $\bullet$ & $\bullet$ & $\bullet$ & $\bullet$ \\
{\bf Introducing bindings} & $\bullet$ & $\circ$ & $\bullet$ & $\circ$ & $\circ$ & $\bullet$ \\
{\bf Depending on bindings} & $\circ$ & $\bullet$ & $\bullet$ & $\circ$ & $\circ$ & $\bullet$ \\
{\bf Compositional} & $\circ$ & $\circ$ & $\circ$ & $\bullet$ & $\bullet$ & $\bullet$ \\
{\bf Analysis of subterms} & $\bullet$ & $\bullet$ & $\bullet$ & $\bullet$ & $\circ$ & $\bullet$  \\
{\bf Constraints on subterms} & $\circ$ & $\circ$ & $\circ$ & $\circ$ & $\circ$ & $\circ$   \\
{\bf Preconditions} & $\bullet$ & $\bullet$ & $\circ$ & $\circ$ & $\circ$ & $\circ$   \\
{\bf Dependencies}  & $\circ$ & $\circ$ & $\bullet$ & $\circ$ & $\bullet$ & $\circ$   \\
{\bf Backwards compatible} & $\bullet$ & $\circ$ & $\bullet$ & $\bullet$ & $\bullet$ & $\bullet$  \\
{\bf Decomposable} & $\circ$ & $\circ$ & $\bullet$ & $\bullet$ & $\bullet$ & $\bullet$  \\ \hline
\end{tabular}
\vspace*{0.5cm}
\newline

\begin{tabular}{rcccccc}
\hline
& {\bf Default parameters} & {\bf Rest parameters} & {\bf Template Literals} & {\bf Generators} & {\bf Let Const} & {\bf Tail call} \\ \hline
{\bf Category} & D. & D. & D. & D. & - & Opt. \\
{\bf Abstraction level} & CfS. & CfS. & CfS. & CfS. & S. & CfS. \\
{\bf Scope} & L2L & L2L & L2L & L2L & G2G & L2L \\
{\bf Extension or Modification} & E. & E. & E. & E. & E. & M. \\
{\bf Syntactically type preserving} & $\bullet$ & $\bullet$ & $\bullet$ & $\bullet$ & $\bullet$  & $\bullet$      \\
{\bf Introducing bindings} & $\circ$ & $\circ$ & $\circ$ & $\circ$ & $\bullet$ & $\circ$ \\
{\bf Depending on bindings} & $\circ$ & $\circ$ & $\circ$ & $\bullet$ & $\circ$ & $\circ$ \\
{\bf Compositional} & $\bullet$ & $\bullet$ & $\bullet$ & $\circ$ & $\circ$ & $\circ$ \\
{\bf Analysis of subterms} & $\circ$ & $\circ$ & $\bullet$ & $\bullet$ & $\bullet$ & $\bullet$ \\
{\bf Constraints on subterms} & $\circ$ & $\circ$ & $\circ$ & $\circ$ & $\circ$ & $\circ$ \\
{\bf Preconditions} & $\circ$ & $\circ$ & $\circ$ & $\bullet$ & $\bullet$ & $\circ$ \\
{\bf Dependencies} & $\circ$ & $\circ$ & $\circ$ & $\circ$ & $\circ$ & $\circ$ \\
{\bf Backwards compatible} & $\bullet$ & $\bullet$ & $\bullet$ & $\bullet$ & $\circ$ & $\bullet$ \\
{\bf Decomposable} & $\circ$ & $\circ$ & $\circ$ & $\circ$ & $\bullet$ & $\circ$ \\ \hline
\end{tabular}
\caption*{$\bullet$: Yes, $\circ$: No, \textbf{D}: Desugaring, \textbf{S}: simplification, \textbf{Opt}: Optimization, \textbf{CfS}: Context-free-syntax, \textbf{L2L}: local-to-local, \textbf{E}: Extension, \textbf{M}: Modification}
\end{table}
\end{landscape}