% Chapter Template

\chapter{Background} % Main chapter title

\label{Chapter2}

\lhead{Chapter 2. \emph{Background}}

\section{ECMAScript}
JavaScript is the programming language implemented from the ECMAScript specification. The specification is maintained by the World wide web consortium (w3c). In this section we will identify some of the key characteristics of the ECMAScript/JavaScript programming language.

\paragraph{Syntax} \label{javascript-syntax}
One of the main concerns for program transformations is syntax. The syntax of the programming language is to be extended to support new language features/constructs. ECMAScript falls in the C-type family % TODO: citation
of syntax definitions. It uses curly braces to delimit blocks of code. 
A program exists of a list of \textit{statements}. A \textit{statement} can be one of the language control from constructs, a function, or an \textit{expression}. Functions in JavaScript can be either defined as an \textit{expression} or as a \textit{statement}\ref{javascript-functions}.

\paragraph{Inhertiance \& Object orientation} \label{javascript-inheritance}
JavaScript can be classified as an object oriented language but without \textit{class} based inheritance model\footnote{Most popular object-oriented languages do have \textit{class} based inheritance (e.g. Java or C++)}. Instead of such a model JavaScript objects have a prototype chain that determine the inheritance. With prototypal inheritance it is possible to implement \textit{class} based inheritance and more.

Everything is an object (with the exception of primitive data-types, e.g. numbers), with properties (even functions \ref{javascript-functions}) and one special property called the \textit{prototype}, this prototype references another object (with yet another prototype) until finally one object has prototype \textit{null}. This chain of objects is called the \textit{prototype chain}. When performing a lookup on a object for a certain key $x$, the JavaScript interpreter will look for property $x$ on our object, if $x$ is not found it will look at the prototype object of our object. This iteration will continue until either prototype \textit{null} is found or an object with property $x$ is found in the prototype chain. 

The best analogy to understand the prototypal inheritance as compared to class based inheritance is that of a buildings blueprint. Every class in class based inheritance is the blueprint for a certain thing/object, when such a class is constructed to an object it is build from this blueprint. And when we call a function on this object it is retrieved from the blueprint. Thus we have an object-class relation. In prototypal inheritance there is no blueprint (or class), but a fully build house (or object). Thus an object-object relation.

Without the use blueprints code will become inefficient rapidly once we start constructing many objects. To avoid this a constructed object has a prototype object which is also a constructed object. The prototype is only constructed once for all the objects that have this object as its prototype.

\paragraph{Scoping} \label{javascript-scoping}
Most programming languages have some form of variable binding. These bindings are only bound in a certain context, such a context is often called a scope. There are two different ways to handle scope in a programming language, Lexical (or static) scoping and dynamic scoping. Lexical scoping is determined by the placing of a binding within the source program and its \textit{lexical context}. Variables bound in a dynamic scope are determined through the program state. 

In JavaScript bindings are determined through the use of lexical scoping, on a function level. This implies that variables bound within a function's context are available throughout the entire execution of the variable. 

One exception on the lexical scoping is the binding of the \textit{this} keyword. Similar to many pure object oriented languages JavaScript makes use of a \textit{this} keyword. In most object oriented languages the current object is implicitly bound to \textit{this} inside functions of that object. \footnote{Sometimes \textit{this} is called \textit{self} (e.g. in the \href{http://ruby.org}{Ruby}) programming language} 
JavaScript being object oriented, binds the \textit{this} variable. In JavaScript the value of \textit{this} can change between function calls. For this reason the \textit{this} binding has dynamic scoping (we can not determine the value of \textit{this} before runtime). 

\paragraph{Typing} \label{javascript-typing}
JavaScript is a dynamically (duck) typed language without type annotations. Each object can be typecast without the use of special operator(s). 

\paragraph{Functions} \label{javascript-functions}
in JavaScript are as explained above also an object. Because of this functions are automatically promoted to first-class citizens (i.e. they can be used anywhere and in anyway normal bindings can be used). This presents the possibility to supply functions as arguments to functions, and calling member functions of the function object. Every function object inherits from the \textit{Function.prototype}\footnote{\url{https://developer.mozilla.org/en-US/docs/Web/JavaScript/Reference/Global_Objects/Function/prototype}} which contains functions that allow the overriding of this binding (see \ref{javascript-scoping}).

Another common pattern is JavaScript related to functions, is that of the immediately invoking function expression (IIFE). Functions in JavaScript can be defined either in a statement or in an expression. When a function is defined as an expression it is possible to immediately invoke the created function:

\begin{lstlisting}[title=IIFE,caption={Immediately invoking function expression}]
(function() {
	<Statement* body>
})()
\end{lstlisting}

\paragraph{Asynchronous}
JavaScript is a single-threaded language, this means that concurrency through the use of processes, threads, or other concurrency constructs is impossible. Many operations that have to wait on some form of IO can be asynchronous (non-blocking). Asynchronous programming is a whole subject of its own and falls outside the scope of this thesis.

\section{Program Transformations}

Eelco Visser defines the aim of a program transformation as follows:

\blockquote[\cite{Visser2001}]{The aim of program transformation is to increase programmer productivity by automating programming tasks, thus enabling programming at a higher-level of abstraction, and increasing maintainability}

There are many different types of program transformations. Programs can be transformed from a source to a target language. Where both can be the same or different. One can be a higher level language and one a lower level language. In this thesis we will focus on transformations in the category rephrasing\cite{Visser2001}. Here a program in a source language is transformed to a different program in the same language. The use case in our thesis is language extensions see section \ref{lang-ext}.

\section{Language Extensions} \label{lang-ext}
\textit{"Language extensions augment a base language with additional language features. Many compilers first desugar a source program to a core language."}\cite{Erdweg2014} For example the Haskell functional programming language defines for many constructs how they can be transformed to a kernel language\cite{PeytonJones}

Language extensions are closely related to other forms of program transformations. There are (syntax) macros\cite{Leavenworth1966}, these are language extensions defined within the same source file and are expanded before compilation (examples SweeterJS\cite{Disney2014}, Syntax macros\cite{Weise1993}). These "define syntactic abstractions over code fragments"\cite{Bravenboer2004}. Macro systems have several limitations that make it difficult to implement a full language extension set, as needed for the transformation of ES6 features. The transformations need to be present in the same source file, macros are often restrained to a fixed invocation (through a macro identifier), and they often lack the expressiveness in their rewriting language\cite{Bravenboer2004}. There was an attempt to implement the ES6 spec through the use of syntax macros but the project\footnote{\url{https://github.com/jlongster/es6-macros}} is abandoned.

Syntactic sugar is another relative of language extensions. As mentioned before there are several languages that before compilation are transformed to a core language (e.g. Haskell core language).  Language features classified as syntactic sugar, are features that can be expressed in the core language. But often benefit from improved human readability when notated with aid of the syntactic sugar. For example the assignment expression $a += b$ is syntactic sugar notation for $a = a + b$ of frequent occurrence in (c-style) programming languages.

\subsection{Hygiene} \label{hygiene}
...

\section{Program Representation} \label{program-representation}
Before programs can be transformed they need a structured representation. Seldom are program transformations performed on the merely the textual input. Here we discuss some program representations.

\paragraph{Parse Trees} 
Parse trees represent a programs syntactic information in a tree. This tree includes layout information of the input program (e.g. white-space, comments). The parse tree is structured according to some context-free grammar, that defines how to parse textual input. The parse tree also deals with disambiguation of input and representation of parentheses. If the parser finds an ambiguity during the parse process the two (or more) alternatives are all inserted in the tree under an ambiguity node.

\paragraph{Abstract Syntax Trees}
or AST is created when layout information is removed from a parse tree and only the core constructs of the language grammar remain as nodes in the tree. White-space layout, comments, and things like parentheses (these are implicitly implied by the structure of the AST) are removed. An AST is often the input for compiler pipelines, program transformations, and refactorings.

\paragraph{Higher Order Syntax Trees (HOAS)}
To represent not just a program's sub expression relations but also its variable binding we can use a Higher Order Syntax Tree (HOAS)\cite{Pfenning}. In a HOAS the variable bindings are made explicit (just as the sub expression relations are made explicit in an AST). Every variable has a binding-site and possibly uses throughout the rest of the tree. \textit{"In addition to dealing with the problem of variable capture, HOAS provides higher-order matching which synthesizes new function for higher-order  variables. One of the problems of higher-order mathcing is that there can be many matches for a pattern"}\cite{Visser2001}

\section{Parsing}
Before a program is represented in on of the previously defined tree formats, it is represented in textual form. To transform the stream of input characters to a tree a program called a parser is used. There are many different types of parser techniques, and a large body of research\cite{Visser1997,Brand2002,Salomon1989}

The parsing of textual input often happens in two stages. First a \textit{scanner} performs the lexical analysis. Which identifies the tokens of our syntax (e.g. keywords, identifiers). With this identification the parser only has to operate on these identified tokens. "The lexical syntax of a language is usually specified by regular expression"\cite{Bravenboer2004}.

Scanners that do not have access to the parser, are also unaware of the context of lexical tokens. In case of JavaScript this can impose subtle difficulties in the implementation of a scanner. "due to ambiguities in how regular expression literals (such as /[0-9]*/) and the divide operator (/) should be lexed."\cite{Disney2014}. For this reason traditional JavaScript parsers are often intertwined with their lexer. Disney et. al. avoid this problem by introducing a separate reader stage in the parser. However this problem can also be avoided by using scannerless parsing, these parsers preserve the lexical structure of the input characters, and make it possible to compose grammars\cite{Visser1997}.  

\section{Language Workbenches} \label{rascal}

The language workbench is a term popularized by Martin Fowler and we can formulate his definition as follows:

A language workbench makes it easy to build tools that match the best of modern IDEs, where the primary source of information is a persistent abstract representation. Its users can freely design new languages without a semantic barrier. With the result of language oriented programming becoming much more accessible.\cite{Fowler2005}

\blockquote[\cite{Fowler2005}]{Essentially the promise of language workbenches is that they provide the flexibility of external DSLs without a semantic barrier. Furthermore they make it easy to build tools that match the best of modern IDEs. The result makes language oriented programming much easier to build and support, lowering the barriers that have made language oriented programming so awkward for so many.}

These workbenches reduce the awkwardness of language oriented programming\cite{Ward1994} significantly, by giving programmers the the tools to define programming languages with an abstract representation.

The RASCAL\cite{Klint} metaprogramming environment (mpl) is a language workbench. It allows programmers to define context-free grammars, generate parsers for these grammars. With these parsers programmers are able to define parse tree patterns. The RASCAL mpl uses Eclipse for IDE integration, which allows programmers to define interaction handles and syntax highlighting. All language extensions presented in this thesis are implemented using the RASCAL mpl.

\paragraph{Syntax definition}
....

\paragraph{Concrete syntax}
A Rascal generated parser returns a parse tree, which is an ordered, rooted tree that represents the syntactic structure of a string according to the formal grammar used to specify the parser. Most transformation systems would implode this concrete syntax tree to an AST and perform program transformations on this tree. Rascal gives the possibility to perform program transformation directly on the concrete syntax tree through the use of concrete-syntax patterns. Presenting multiple advantages over an AST based solution:
\begin{itemize}
	\item Preserving layout information encoded in the concrete-syntax tree
	\item Avoiding the use of a pretty printer to output the transformed AST to a textual representation
	\item Transformation code reads as rewrite rules, instead of a clutter of AST nodes. (Concrete syntax is syntax highlighted using the generated parser)
\end{itemize}
A concrete-syntax pattern may contain variables in the form of our grammars non-terminals, these variables are bound in the context of the current patterns use (e.g. functions body when used as a formal parameter). The following pattern matches unnamed JavaScript functions according to the Saner JavaScript syntax definition\footnote{\url{https://goo.gl/B6HR22}}

\begin{lstlisting}
(Expression)`function( <Params ps> ) { <Statement* body> }` := pt
\end{lstlisting}

In similar fashion new syntax trees can be constructed, using the variables bound from pattern matches.

\begin{lstlisting}
(Expression)`function plus(x, y) { return x + y; }`
\end{lstlisting}

Rascal also supplies a shorthand notation to parse a string starting from the supplied non-terminal

\begin{lstlisting}
[Expression]"1 + 2"
\end{lstlisting}