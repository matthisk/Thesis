% !TEX root = ../main.tex
% Chapter Template

\chapter{Problem Analysis} % Main chapter title

\label{Chapter2}

\lhead{Chapter 2. \emph{Problem Analysis}}

\section{Motivation}
What are the reasons to create program transformations or extend current programming languages, as opposed to just creating entirely new programming languages? Extensible programming languages allow programmers to introduce custom language features as an extension on a base language.  There are two main reasons for programmers to use program transformations. \textit{``First, a programmer can define language extensions for language constructs that are missing in the base language''}~\cite{Erdweg2014a}, for example default parameters, type annotations, or list comprehensions in JavaScript. \textit{``Second, extensible programming languages serve as an excellent base for language embedding''}~\cite{Erdweg2014a} this refers to the embedding of domain-specific languages inside a host language, for example using HTML markup inside of JavaScript with JSX\footnote{\url{https://facebook.github.io/jsx/}}.

The implementation of \projectname{} focuses on the first motivation for the extension of programming languages, to introduce missing language constructs to a base language, because the ES6 specification only introduces new language constructs and does not embed any domain-specific languages in JavaScript. Similar techniques as to those used to implement \projectname{} can be used for the embedding of domain-specific languages in a host language, our research thus applies to both motives. We will evaluate the language workbench (see section \ref{rascal}) for the task of creating such language extensions. With the help of an evaluation of our language extension set we will measure the effectiveness of the language workbench along several different dimensions:

\begin{itemize}
	\item Difficulty of implementing language extensions
	\item Ease with which language extensions integrate with the development environment of the programmer
	\item Modularity of language extensions, i.e. is it easy or hard to add new language extensions on top of each other and on top of the base language.
\end{itemize}

JavaScript is an interpreted programming language (see section \ref{javascript}) used to run code inside web-browsers to power interactivity of websites. JavaScript programmers are forced to use the JavaScript version implemented in the majority of their users web-browsers. Because of this they can not benefit from language features introduced by new versions until most of their users have updated their browsers to a version implementing this new standard. Language extensions can solve this issue by transforming the new language constructs to constructs available in the majority of their users web-browsers.

\section{Research approach}
Our research exists out of an experiment and an evaluation. The experiment is an implementation of a subset of the ES6 specification as language extensions on top of ES5 JavaScript as a base language, this implementation is our research's artifact (see appendix \ref{AppendixA}), named \textit{\projectname}. It is implemented in the Rascal programming language (see section \ref{rascal}). Second, we evaluate the language workbench for the task of implementing language extensions. For this evaluation we compare our implementation to other transformers that transform ES6 JavaScript to ES5 JavaScript.

\section{Research questions}
The central research questions of this thesis are:

\begin{enumerate}
	\item How \textit{effective} is the language workbench for the task of implementing language extensions?
	\item How can program transformations be categorized and give insight into differences between language extensions?
\end{enumerate}

Where effectiveness is established along several dimensions, ease with which the extensions can be implemented (i.e. time and source lines of code needed for transformation code), modularity of implementation i.e. is it easy or hard to add new language extensions on top of each other and the base language. With the effectiveness established we determine the engineering trade-offs of using the language workbench for implementation of language extensions.

\section{Results}
\begin{itemize}
	\item A taxonomy for language extensions
	\item Language extension suite implementing (a subset of) the ES 6 features
	\item A set of compatibility tests, testing our implementation against the specification document
	\item Eclipse IDE integration (syntax highlighting \& declared at hyperlinks) for new language features
	\item Set of engineering trade-offs applicable to implementing language extensions inside the language workbench
\end{itemize}

A set of language extensions for the JavaScript programming language created with the aid of a language workbench. First, we expect that the use of language workbench will reduce the amount of work needed to implement these language extensions. Second, we expect that the use of a language workbench will make our implementation more modular, making it easier to add additional language extensions on top of other extensions or on top of the base language. 

The taxonomy of language extensions and the categorization created with the help of this taxonomy will provide us with useful insights in the inner working of these language extensions as program transformations. And will help in establishing a view of the key characteristics of each language extension.

We expect to get insight into the engineering trade-offs of implementing language extensions inside the language workbench. What limitations does the workbench impose, where can we benefit from the power of a language workbench, and what are the generic problems of language extensions.

\section{Related Work} \label{sec:related}
Many researchers have investigated the possibility to make programming languages extensible~\cite{Erdweg2014a,Bravenboer2004}, be it through language extensions, compiler extensions~\cite{Zenger2001,Nystrom2003}, or syntax macros~\cite{Disney2014,Kohlbecker1986,Leavenworth1966,Weise1993}. Often their research also tries to extend the tools with which programmers write their code (i.e. IDEs). Here we discuss several different approaches to programming language extension presented in the past.

Object oriented programming languages are designed in such a way that libraries created in the language encapsulate knowledge (in semantics) of a specific domain for which the library is created. Programmers of such libraries are not allowed to encapsulate knowledge of the domain in syntax. To allow the extension of programming languages to encapsulate domain knowledge in syntax Eelco Visser created a system for language extension through concrete syntax named MetaBorg~\cite{Bravenboer2004}. The system is realized through Syntax Definition Formalism~\cite{Heering1989} and the Stratego~\cite{Visser2001a} transformation language. The system is modular and composable this makes it possible to combine different language extensions. The paper discusses three domains which \textit{``suffer from misalignment between language notation and the domain: code generation, XML document generation, and graphical user-interface construction.''}~\cite{Visser20024} To test MetaBorg, DSLs for these three domains are implemented and integrated with the Java programming language as a host language. The work of Visser and Bravenboer relates closely to \projectname{} presented in this thesis. Visser ignores the issues related to introduction of new bindings (i.e. variable capture). MetaBorg is not evaluated by quantitative measures against other tools for language extension.

Another system based on SDF and Stratego is presented by Erdweg et. al.~\cite{Erdweg} called Sugar*. This is a system for language extension agnostic from the base language, the system can extend- syntax, editor support, and static analysis. In contrast to the system presented in this thesis, Sugar* relies on information from a compiler of the base language to operate correctly. To evaluate the Sugar* system measures based on lines of code are applied to language extensions created for five different base languages (Java, Haskell, Prolog, JavaScript, and System F$_{\omega}$). Sugar* uses Spoofax to create editor support for language extensions, similar to a language workbench. Sugar* is based on SugarJ~\cite{Erdweg2011}, a library based language extension tool for the Java programming language. SugarJ is evaluated against other forms of syntactic embedding on five dimensions defined as design goals for SugarJ. They compare against string encoding, pure embedding, macro systems, extensible compilers, program transformations, and dynamic meta-object protocols.

Importing multiple libraries that include new syntax constructs could result in syntactic ambiguities among the new constructs. The Sugar* and SugarJ framework have no way to resolve these ambiguities and clients of the libraries will have to resolve these ambiguities themselves, requiring a reasonable thorough understanding of the underlying parser technology~\cite{Omar2014}. Omar et. al.~\cite{Omar2014} sidestep this problem in their research by introducing an alternative parsing strategy. Instead of introducing syntax extensions on top of the entire grammar, extensions are tied to a specific type introducing \textit{type-specific languages} (TSL). TSL's are created with the use of layout sensitive grammars, presenting a new set of challenges not faced in this thesis because we do not rely on layout sensitive grammars. The paper introduces a theoretical framework for TSL's and an implementation named Wyvern. A corpus of 107 Java projects is analyzed to assess how frequently TSL's could be introduced instead of Java constructors. The work lacks any empirical validation of introduced tooling.

All the tools discussed above (MetaBorg, Sugar*, and SugarJ) rely for their evaluation on a very small subset of actual implemented language extensions, where most are based on embedding a domain-specific language in a host language. In this thesis we try to evaluate the language workbench for the job of language extensions with the help of a large set of implemented extensions not based on embedding domain-specific languages but introducing new language features.  

The extension of the JavaScript programming language is studied by Disney et. al. ~\cite{Disney2014}, here JavaScript is extended through the use of syntax macros instead of separate language extensions. Their work focuses more on separating the parser and lexer to avoid ambiguities during parsing (a problem we avert by using a scannerless parser see section \ref{parsing}). They give no evaluation of the expressiveness of the resulting macro system. A project trying to implement ES6 features as macros using their macro system exists but has been abandoned\footnote{\url{https://github.com/jlongster/es6-macros}}

In related work of our taxonomy of language extensions (see chapter \ref{Chapter4}) we can identify several categorizations of program transformations. The Irvine program transformation catalog~\cite{Standish1976} present a set of useful program transformations for lower-level procedural programming languages (e.g. the C programming language). Visser~\cite{Visser2001} presents a taxonomy for program transformations, which we partly reuse in our taxonomy of language extensions. This taxonomy can be used for program transformations in general where our taxonomy is used to categorize language extensions, which can be implemented with the use of program transformations but are not limited to them.

Erdweg et. al.~\cite{Erdweg2014} present the \textit{name-fix} algorithm, a generic language parametric algorithm that can solve the issue of variable capture that arises after program transformations stand-alone from transformation code. In this thesis we reuse a similar system to ensure no variable capture arises during transformation. Erdweg et. al. prove the correctness of their implementation, something we have not performed for the implementation we reuse. A survey of domain-specific language implementations gives valuable insight into variable capture to be found in program transformations. Transformation hygiene has also received a lot of attention in the domain of macros~\cite{Kohlbecker1986,Herman2010a,Disney2014}. 
