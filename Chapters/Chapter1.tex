% !TEX root = ../main.tex
% Chapter 1

\chapter{Introduction and Motivation}

\label{Chapter1}

\lhead{Chapter 1. \emph{Introduction}}

Language extension allow programmers to introduce new language constructs to a base language, with two main purposes. \textit{"First a programmer can define language extensions for language constructs that are missing in the base language"}~\cite{Erdweg2013}. \textit{"Second, extensible programming languages serve as an excellent base for language embedding"}~\cite{Erdweg2013}. Language extensions are eliminated from the source program with the use of program transformations. Many systems for program transformation exist but in recent years a specific type of tool has become more popular for this job, the language workbench. This tool aids the meta-programmer in creating programs that manipulate other programs while integrating with the developer tools of the programmer. In this thesis we investigate the ability of language workbenches in helping the meta-programmer to create a large set of language extensions. 

As an experiment we extend the JavaScript programming language with features introduced by the new specification document of the language, ECMAScript 6 (ES6). The current specification of JavaScript implemented in all major run-times (be it web-browser or dedicated) is ECMAScript 5 (ES5). The language extensions are created with the Rascal~\cite{Klinta} language workbench and the resulting tool is named \projectname. A second contribution of this thesis is a taxonomy for language extensions. With this taxonomy we try to capture the distinctive characteristics of each language extension in a generic way. In the appendix we present a full categorization of ES6 language extensions according to this taxonomy. 

Because of the popularity of the JavaScript programming language there already exist several implementations of ES6 as language extensions for ES5 JavaScript. This presents us the opportunity to evaluate the language workbench for the task of extending programming languages against language extensions implemented without the help of a language workbench.  
Measures used to evaluate our implementation are based on lines of code, correctness of the transformations, noise generated by the transformation, coverage of the transformation suite, and modularity of the language extensions.

The language workbench made it possible for us to implement ES6 language extensions in a short time period with less lines of code. We cover almost all new language features from ES6 in \projectname, something only other large-scale open-source projects are able to achieve. We deliver editor support for reference hyperlinking, undeclared reference errors, illegal redeclaration errors, and hover documentation preview of target program. The language workbench does however constraint us to one specific IDE and our solution is less portable than other implementations. Syntax definition of the language workbench constrainted us from implementing certain features of the ES6 and ES5 grammar.

\section{Outline}
This thesis is structured as follows. In chapter \ref{Chapter2} we present an analysis of the problem studied in this thesis. Chapter \ref{Chapter3} presents background information of program transformations and the language workbench. In chapter \ref{Chapter4} we present a taxonomy for language extensions.  Chapter \ref{Chapter5} discusses the implementation of \projectname. In chapter \ref{Chapter6} we evaluate our implementation against other implementations. Finally we conclude in chapter \ref{Chapter7}.