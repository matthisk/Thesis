% Chapter 1

\chapter{Introduction and Motivation}

\label{Chapter1}

\lhead{Chapter 1. \emph{Introduction and Motivation}} 

\section{Context}
JavaScript is an implementation of the ECMAScript specification. Since the release of the first version of the specification there have been four iterations (where version four was skipped), the current version of the specification is ECMAScript 5. In the near future a new version of the specification will leave the draft status and become the standard (ECMAScript 6). With a new specification come new syntactic language features. Before these features can be used in the JavaScript run-time environment of choice, the vendors of these environments have to implement the new standard. However it is possible to use the new standard today through the use of program transformations.

In this work we study the extension of programming languages, the new ECMAScript specification presents an opportunity to research these extensions and there transformations. Using the RASCAL meta-programming environment\citep{Klint} we will implement the program transformations. Afterwards we can perform an analysis on the resulting transformation suite to uncover the \emph{effectiveness} of our implementation.

\textit{"Program transformations find ubiquitous application in compiler construction to realize desugaring, optimizes, and code generators"}\cite{Erdweg2014} 
in our research we will focus on the desugaring of (new) language constructs.  

\textit{"The aim of program transformation is to increase programmer productivity by automating programming tasks, thus enabling programming at a higher-level of abstraction, and increasing maintainability"}\cite{Visser2001}

\section{Problem definition}
Program transformations are faced by multiple problems, in what order do we run the transformation rules, how do we represent the program (e.g. abstract syntax) to perform transformations on, what guarantee do we have for the validity of our transformations, how can a transformation introduce new bindings.  
Some of these problems reoccur in every transformation that is created. These problems are identified as cross-cutting concerns. The goal of this thesis is to identify these concerns and separate them from the transformation code. 

The capture avoidance is one such problem, it is mostly studied in the context of macro expansion\cite{Herman2010a,Herman2010,Disney2014}. And is often called the hygiene of transformations. When an identifier originating from the source program references to a synthesized declaration, the transformation is sub-hygienic and introduces variable capture. To prevent this problem we implement the \textit{name-fix} algorithm\cite{Erdweg2014} for our transformation suite.

\section{Scope}

\subsection{Expected results}
This thesis will have the following results:
\begin{itemize}
	\item A taxonomy of ES6 language features specified by their needed transformations
	\item Language extension suite implementing (a subset of) the ES6 features, which is less verbose than current implementations (i.e. BabelJS\footnote{\url{http://babeljs.io}} and Traceur\footnote{\url{https://github.com/google/traceur-compiler}}).
	\item Identification of cross-cutting concerns of our language extension suite
	\item A set of compatibility tests, testing our implementation against the specification document.
	\item Eclipse integration (syntax highlighting \& declared at hyperlinks) for new language features
\end{itemize}
\section{Research questions}
The central research questions of this thesis are:

\begin{enumerate}
	\item How can independent program transformations be categorized?
	\item What are the cross-cutting concerns of any program transformation, and how can we take care of these concerns outside of the transformations themselves?
	\item What is the advantage of using the RASCAL language workbench\cite{Klint} for the creation of our transformation suite?
\end{enumerate}

\section{Outline}
In Chapter \ref{Chapter2} the background (program transformations, language workbench) is introduced. In Chapter \ref{Chapter3} we give an overview of language extensions that are introduced and make a taxonomy of their transformations. In \ref{Chapter4} we discuss the cross-cutting concerns of the transformation suite (i.e. variable capture). In Chapter \ref{Chapter5} we evaluate our resulting transformation suite (against others). Finally, we conclude in Chapter \ref{Chapter6}.