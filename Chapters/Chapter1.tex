% !TEX root = ../main.tex
% Chapter 1

\chapter{Introduction and Motivation}

\label{Chapter1}

\lhead{Chapter 1. \emph{Introduction}}

\textit{"Program transformations find ubiquitous application in compiler construction to realize desugaring, optimizers, and code generators"}~\cite{Erdweg2014}. A specific type of program transformation is that of a language extension. Such an extension introduces a new construct to the base language with a desugaring step to translate the construct to a fundamental concept of the base language. Language extensions serve two main purposes. \textit{"First a programmer can define language extensions for language constructs that are missing in the base language"}~\cite{Erdweg2013}. \textit{"Second, extensible programming languages serve as an excellent base for language embedding"}~\cite{Erdweg2013}. 

In recent years domain-specific tools for the meta-programmers have become popular. Tools that aid the meta-programmer in creating programs that manipulate other programs while integrating with the developer tools of the programmer are called language workbenches. In this thesis we investigate the ability of language workbenches in helping the meta-programmer to create a large set of language extensions. 

As an experiment we extend the JavaScript programming language with features introduced by the new specification document of the language, ECMAScript 6 (ES6). The current specification of JavaScript implemented in all major run-times of JavaScript (be it web-browser or dedicated) is ECMAScript 5 (ES5). The language extensions are created inside the Rascal~\cite{Klinta} language workbench and the tool presented in this thesis is called \projectname. Before implementation of \projectname we examined all the needed language extensions and created a categorization following a taxonomy created in this thesis for the categorization of language extensions. This taxonomy can give programmers valuable insights into the intricacies of implementing a language extension.

Because of the popularity of the JavaScript programming language there already exist several implementations ES6 as language extensions for ES5 JavaScript. This presents us the opportunity to evaluate the language workbench for the task of extending programming languages against language extensions implemented with the absence of a language workbench.  
Measures used to evaluate our implementation are based on lines of code of transformation code, correctness of the transformations, noise generated by the transformation, coverage of the transformation suite, and modularity of the language extensions.

The language workbench made it possible for us to implement ES6 language extensions in a short time period with less lines of code. We cover almost all new language features from ES6 in \projectname, something only other large-scale open-source projects are able to achieve. The language workbench does however constraint us to one specific IDE and our solution is less portable than some other implementations.

\section{Outline}
In \ref{Chapter2} we present an analysis of the problem studied in this thesis. Chapter \ref{Chapter3} presents background information of program transformations and the language workbench. In chapter \ref{Chapter4} we present a taxonomy for language extensions.  Chapter \ref{Chapter5} discusses the implementation of \projectname. In chapter \ref{Chapter6} we evaluate our implementation against other (open-source) implementations. Finally we conclude in chapter \ref{Chapter7}.